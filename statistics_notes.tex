\documentclass[twoside, a4paper, 10pt]{amsart}
\title[ ]{Mathematical Statistics}
%\usepackage{amsaddr}
%\email{Kamil.Bulinski@minetec.com.au}
\usepackage{amsfonts}
\usepackage{amsthm}
\usepackage{verbatim}
\usepackage{amsmath, amssymb}
\usepackage{tikz}
\usetikzlibrary{matrix, arrows}
\usepackage{listings}
\usepackage{color}
\usepackage{listings}
\usepackage[all]{xy}
\usepackage[pdftex,colorlinks,linkcolor=blue,citecolor=blue]{hyperref}
\usepackage{graphicx}
\usepackage{float}
\usepackage[margin=3cm]{geometry}
\usepackage{bigints}
\usepackage{dsfont}
\setlength{\textwidth}{6.5in}
\setlength{\oddsidemargin}{0in}
\setlength{\evensidemargin}{0in}
\setlength{\parindent}{0pt}
\setlength{\parskip}{1ex plus 0.5ex minus 0.2ex}
\linespread{1.3}


\setcounter{secnumdepth}{4}

\begin{document}
\maketitle
\raggedbottom


%% Mathcal large
\newcommand{\cA}{\mathcal{A}}
\newcommand{\cB}{\mathcal{B}}
\newcommand{\cC}{\mathcal{C}}
\newcommand{\cD}{\mathcal{D}}
\newcommand{\cE}{\mathcal{E}}
\newcommand{\cF}{\mathcal{F}}
\newcommand{\cG}{\mathcal{G}}
\newcommand{\cH}{\mathcal{H}}
\newcommand{\cI}{\mathcal{I}}
\newcommand{\cJ}{\mathcal{J}}
\newcommand{\cK}{\mathcal{K}}
\newcommand{\cL}{\mathcal{L}}
\newcommand{\cM}{\mathcal{M}}
\newcommand{\cN}{\mathcal{N}}
\newcommand{\cO}{\mathcal{O}}
\newcommand{\cP}{\mathcal{P}}
\newcommand{\cQ}{\mathcal{Q}}
\newcommand{\cR}{\mathcal{R}}
\newcommand{\cS}{\mathcal{S}}
\newcommand{\cT}{\mathcal{T}}
\newcommand{\cU}{\mathcal{U}}
\newcommand{\cV}{\mathcal{V}}
\newcommand{\cW}{\mathcal{W}}
\newcommand{\cX}{\mathcal{X}}
\newcommand{\cY}{\mathcal{Y}}
\newcommand{\cZ}{\mathcal{Z}}
%% Mathbb large
\newcommand{\bA}{\mathbb{A}}
\newcommand{\bB}{\mathbb{B}}
\newcommand{\bC}{\mathbb{C}}
\newcommand{\bD}{\mathbb{D}}
\newcommand{\bE}{\mathbb{E}}
\newcommand{\bF}{\mathbb{F}}
\newcommand{\bG}{\mathbb{G}}
\newcommand{\bH}{\mathbb{H}}
\newcommand{\bI}{\mathbb{I}}
\newcommand{\bJ}{\mathbb{J}}
\newcommand{\bK}{\mathbb{K}}
\newcommand{\bL}{\mathbb{L}}
\newcommand{\bM}{\mathbb{M}}
\newcommand{\bN}{\mathbb{N}}
\newcommand{\bO}{\mathbb{O}}
\newcommand{\bP}{\mathbb{P}}
\newcommand{\bQ}{\mathbb{Q}}
\newcommand{\bR}{\mathbb{R}}
\newcommand{\bS}{\mathbb{S}}
\newcommand{\bT}{\mathbb{T}}
\newcommand{\bU}{\mathbb{U}}
\newcommand{\bV}{\mathbb{V}}
\newcommand{\bW}{\mathbb{W}}
\newcommand{\bX}{\mathbb{X}}
\newcommand{\bY}{\mathbb{Y}}
\newcommand{\bZ}{\mathbb{Z}}


\newcounter{dummy} \numberwithin{dummy}{section}

\theoremstyle{definition}
\newtheorem{mydef}[dummy]{Definition}
\newtheorem{prop}[dummy]{Proposition}
\newtheorem{corol}[dummy]{Corollary}
\newtheorem{thm}[dummy]{Theorem}
\newtheorem{lemma}[dummy]{Lemma}
\newtheorem{eg}[dummy]{Example}
\newtheorem{notation}[dummy]{Notation}
\newtheorem{remark}[dummy]{Remark}
\newtheorem{claim}[dummy]{Claim}
\newtheorem{Exercise}[dummy]{Exercise}
\newtheorem{question}[dummy]{Question}
\newtheorem{conjecture}[dummy]{Conjecture}

\section{Basic Setup}


\begin{mydef}[Statistical setup] A \textit{statistical setup} consists of the following data:

We have the following measurable spaces:

\begin{itemize}
	\item The \textit{observation space} $\cX$.
	\item The \textit{parameter space} $\Theta$.
	\item The \textit{action space} $\cA$.
	\item The \textit{sample space} $\Omega$
\end{itemize}

We also have the following maps:

\begin{itemize}
	\item A \textit{loss function} $L:\Theta \times \cA \to \bR$.
	\item An \textit{observation} $X: \Omega \to \cX$.
	\item A \textit{decision function} $d:\cX \to \cA$.

\end{itemize}

For each parameter $\theta \in \Theta$, we have a probability measure $P_{\theta} \in \operatorname{Prob}(\Omega)$. The associated risk function is then \begin{align*} R(\theta, d) &= E_{\theta} L(\theta, d(X)) \\ &= \int_{\Omega} L(\theta, d(X(\omega)) dP_{\theta}(\omega).\end{align*}

\end{mydef}

Note: It seems that unless otherwise specified, it does not hurt to just assume $\cX = \Omega$ and $X$ is the identity map in which case $$R(\theta, d) = \int_{\cX} L(\theta, d(x)) dP_{\theta}(x).$$

\textbf{Interpretation:} We have an unknown parameter $\theta$ and a given observation $X(\omega) \in \cX$ where $\omega \in \Omega$ was according to $P_{\theta}$. Based on this observation, we take an action $d(x) \in \cA$. Then the loss $L(\theta, d(x))$ measures how wrong we are in taking this action $d(x)$. The risk $R(\theta, d)$ is then the average of this loss (for the fixed $\theta$).

\begin{eg} Suppose now $\Omega = \cX = \bR^n$ and $X$ is identity (as usual) and $\cA = \bR$. Let $\Theta = \bR \times \bR_{\geq 0}$ and for $\theta = (\mu, \sigma^2) \in \Omega$ we let $P_{\theta}$ denote the distribution of the random variable $(X_1, \ldots, X_n) \in \bR^n$ where the $X_i$ are i.i.d $N(\mu, \sigma^2)$. We take a loss function $L(\theta, a) = |\mu - a|^2$. Thus if $d:X \to \cA$ is a decision rule then $L(\theta, d(x))$ will measure how well of an approximation $d(x) \in \bR$ is of the unknown $\mu$. Let us choose $d(x_1, \ldots, x_n) = \frac{1}{n} \sum_{i=1}^n x_i$. Thus $$R(\theta, d) = E |\frac{1}{n}\sum_{i=1}^N X_i - \mu|^2 = Var(\frac{1}{n}(X_1 + \ldots + X_n)) = \frac{\sigma^2}{n}$$ by the independence of the $X_i$. We see that as $R(\theta, d) = O(n^{-1})$ so this seems to be a good decision rule.

If instead we use the loss function $L(\theta, a) = |\sigma^2 - a|$ then a good decision rule seems to be $$d(X) = s^2 := \frac{1}{n-1} \sum_{i=1}^n |X_i - \overline{X}|^2.$$ It turns out that (TODO?) $$R(\theta, d) = \frac{2\sigma^4}{n-1}.$$

\end{eg} 

We can use this language to define hypothesis testing:

\begin{mydef} Suppose now that we have a partition $\Theta = \Theta_0 \sqcup \Theta_1$ of the parameter space and our action space is $\cA = \{ \Theta_0, \Theta_1\}$. We have a $0-1$ loss function given by $$L(\theta, \Theta_i) = 1 - \mathds{1}_{\Theta_i}(\theta).$$ Thus $L(\theta, a)$ measures whether we chose the right partition $a \in \cA$ that $\theta$ landed in. We see that the risk $$R(\theta, d) = P_{\theta} (\left\{ \omega \in \Omega ~|~ \theta \notin d(X(\omega)) \right\} )$$ is the probability that our decision algorithm chooses the wrong partition.

The action $\Theta_0$ is called the \textit{Null Hypothesis} while $\Theta_1$ is called the \textit{Alternative hypothesis}. A \textit{Type I} error is an action that chooses the alternative hypotehesis ($\Theta_1$) when the null hypotheiss is true ($\theta \in \Theta_0$). A \textit{Type II error} is an action that chooses the null hypothesis when the alternative is true.

\end{mydef}

\begin{eg} Suppose we have a dice that is either fair (null hypothesis) or always rolls a 6 (alternative hypothesis). Suppose we have rolled this dice $n$ times and observed the outcomes $(X_1, \ldots, X_n) \in \{1,2,3,4,5,6\}^n = \cX$. Say $\Theta = \{\theta_0, \theta_1\}$ where $P_{\theta_0}$ is the uniform distribution on $\cX$ (dice is fair) while $P_{\theta_1}$ is the probability measure concentrated on $(6, \ldots, 6) \in \cX$. Our decision rule simply chooses $\theta_1$ if and only if we roll all $6$. We have $R(\theta_0, d) = 1/6^n$ while $R(\theta_1,d) = 0$.

\end{eg}

\begin{mydef} \label{def: nice} We say that a statistical setup is \textit{nice} if there exists a positive measure (not necessarily a probability measure) $\nu$ on $\Omega = \cX$ and a measurable function $f:\cX \times \Theta \to \bR_{\geq 0}$ such that $P_{\theta} \ll \nu$ and $$\frac{dP_{\theta}}{d\nu} (x) = f(x|\theta) := f(x, \theta)$$ for all $\theta \in \Theta$.

\end{mydef}

\textbf{Assumption going forward:} We always assume a statistical setup is nice without mentioning it. This is the case for the examples above (i.e., $\nu$ is either a Lebesgue measure or a counting measure).

\section{Bayesian Setup}

\begin{mydef} A \textit{prior distribution} is a probability measure $\mu_{\Theta}$ on the parameter space $\Theta$. \end{mydef}

Recalling that we assume the nice condition of Definition~\ref{def: nice}, we thus have a probability measure $P$ on $\cX \times \Theta$ given by $$\frac{dP}{d(\nu \times \mu_{\Theta})}(x,\theta) = f(x|\theta).$$ In other words for integrable $\psi:\cX \times \Theta \to \bR$ we have 

\begin{align*} \int_{\cX \times \Theta} \psi(x,\theta) dP(x,\theta) &= \int_{\cX \times \Omega } \psi(x,\theta) f(x|\theta) d(\nu \times \mu_{\Theta}) (x,\theta) \\ &= \int_{\Theta} \left( \int_{\cX} \psi(x,\theta) f(x|\theta) d\nu(x) \right) d\mu_{\Theta}(\theta) \\
&=  \int_{\Theta} \int_{\cX} \psi(x,\theta) dP_{\theta}(x) d\mu_{\Theta}(\theta)   \end{align*}

From now on, we let $X:\cX \times \Theta \to \cX$ and $\Theta:\cX \times \Theta \to \Theta$ denote the push-forward measure (note: by abuse of notation $\Theta$ is now either a space or a projection map, but it should cause no confusion in any context).

Let $P_X=X^*P$ denote the push-forward measure of $P$ onto $\cX$, that is, $P_X(A) = P(A \times \Theta)$ for $A \subset \cX$. We let $$g(x) := \int_{\Theta} f(x,\theta) d\mu_{\Theta}(\theta).$$

Observe that for $A \subset \cX$ we have that $$P_X(A) = \int_{A \times \Theta} f(x|\theta) d\mu_{\Theta}(\theta) d\nu(x) = \int_{A} g(x) d\nu(x)$$ and thus $P_X$ has a density with respect to $\nu$ given by $g$.
\begin{lemma} For $P_X$-almost all $x \in \cX $ we have that $$g(x)  > 0.$$

\end{lemma}

\begin{proof} Let $A = \{x \in \cX ~|~ g(x) = 0 \}$. Then $$P_X(A) = \int_{A} g(x) d\nu(x) = 0.  $$

\end{proof}

Given this lemma, we can now define for $P_X$-almost all $x \in \cX$ a probability measure on $\Theta$ given by $$P_{\Theta|X=x}(B) = \frac{1}{g(x)} \int_{B} f(x|\theta) d\mu_{\Theta}(\theta)$$ for $B \subset \Theta$. This is known as the \textit{posterior distribution of $\Theta$ given $X = x$}.

\begin{lemma} The measures $\delta_x \times P_{\Theta |X=x}$ are conditional distributions of $P$ given $X=x$ where $X:\cX \times \Theta \to \cX$ is the projection. The conditional expectation of an integrable $\psi:\cX \times \Theta \to \bR$ given $X=x$ is given by $$E(\psi | X=x) = \frac{1}{g(x)}\int_{\Theta} \psi(x,\theta) f(x|\theta) d\mu_{\Theta}(\theta).$$ 

\end{lemma}

\begin{proof} Let $E'(\psi |X=x)$ denote the right hand side. We first show that it is well defined for almost all $x \in X$. This follows since by assumption we have that $|\psi(x,\theta)|$ is integrable with respect to $P$ and thus $$\int |\psi(x,\theta)| f(x|\theta) d\nu(x) d\mu_{\Theta} (\theta) < \infty.$$ It follows from Fubini-Tonelli that for $\nu$ almost all $x \in X$ we have that $$\int_{\cX} |\psi(x,\theta)| f(x|\theta) d\mu_{\Theta}(\theta) < \infty$$ and thus $E'(\psi | X=x)$ is a well defined measurable function on $X$. We have to show that $(x, \theta) \mapsto E'(\psi |X=x)$ is measurable with respect to the $\sigma$-algebra induced by $X: \cX \times \Theta \to \cX$ and that for all $A \subset X$ we have that $$\int_{A \times \Theta} E'(\psi |X=x) dP(x, \theta) = \int_{A \times \Theta} \psi(x,\theta) dP(x, \theta).$$

The measurability is obvious as there is no dependence on $\Theta$. For the second claim, note that since (as shown above) $g = dP_X/d\nu$, we have 

\begin{align*} \int_{A \times \Theta} E'(\psi |X=x) dP(x, \theta) &= \int_{A} E'(\psi |X=x) dP_X(x) \\
&=\int_{A} E'(\psi |X=x) g(x) d\nu(x) \\ 
&= \int_{A} \int_{\Theta} \psi(x, \theta) f(x|\theta) d\mu_{\Theta} d\nu(x)\end{align*}

which completes the proof of this property since $f(x|\theta) = \frac{dP}{d(\nu \times \mu_{\Theta})}(x,\theta) $. Finally, observe that $$\int _{\cX \times \Theta} \psi ~d(\delta_x \times P_{\Theta|X=x}) = \int_{\Theta} \psi(x,\theta) dP_{\Theta |X=x}(\theta) = \frac{1}{g(x)} \int_{\Theta} f(x|\theta) \psi(x,\theta) d\mu_{\Theta}(\theta) = E(\psi ~|~ X=x) $$ so indeed $\delta_x \times P_{\Theta|X=x}$ is the conditional probability of $P$ given $X=x$. \end{proof}

By symmetry, if we swap $\cX$ with $\Theta$ then we get 

\begin{align*} P(A \times \Theta ~|~ \Theta = \theta) &= \left(\int_{\cX} f(x|\theta) d\nu(x) \right)^{-1} \left(\int_{A} f(x|\theta) d\nu(x) \right) \\
&= P_{\Theta}(X)^{-1} P_{\theta}(A) \\
&= P_{\theta}(A) \end{align*}

\subsection{Notation} We use the standard notation from probability theory: if $F_i:\Omega \to \Omega_i$ are measurable and $P$ is a measure on $\Omega$, then $$P(F_1 \in A_1, \ldots, F_n \in A_n) := P(\{\omega \in \Omega ~|~  F_1(\omega) \in A_1, \ldots, F_n(\omega) \in A_n \})$$ and likewise for conditional measures.

Thus if $P$ is the measure on $\cX \times \Omega$ as constructed above, then $$P(X \in A, \Theta \in B) = P(A \times B).$$ 

\section{Sufficient statistics}

\subsection{Basic definitions and examples}

A statistic is a measurable function $T:\cX \to \cT$ to some measurable space $(\cT, \cB_{\cT})$.  

\begin{mydef} We say that a statistic $T$ on $\cX$ is \textit{sufficient (in the Bayesian sense)} if for any prior $\mu_{\Theta}$ on $\Theta$ we have for any Borel $B \subset \Theta$ the equality $$P(\Theta \in B ~|~ X = x) = P(\Theta \in B ~|~ T \circ X = Tx) \quad \text{ for } \mu_X\text{-almost all } x \in \cX.$$

\end{mydef}


\begin{eg} \label{eg: bernoulli bayes setup} Consider the case where $\cX = \{0,1\}^n$ and $\Theta = [0,1]$. Suppose that $P_{\theta}$ is the Binomial probability distribution on $\{0,1\}^n$ where $(X_1, \ldots, X_N) \in \cX$ is drawn i.i.d with $P(X_i = 1) = \theta.$ Let us for now assume the Prior distribution $\mu_{\Theta}$ is unknown. We know that $$P_{\theta}(X = x) =   \theta^{T(x)}(1-\theta)^{n - T(x)} $$ where $T(x) = \sum_{i=1}^n x_i$ is the number of successes. The posterior distribution is thus given by $$P(\Theta \in B ~|~ X=x) = \frac{P(\Theta \in B, X=x)}{P(X=x)} = \frac{ \int_B \theta^{T(x)}(1-\theta)^{n - T(x)} d \mu_{\Theta}(\theta)}{\int_{\Omega} \theta^{T(x)}(1-\theta)^{n - T(x)} d \mu_{\Theta}(\theta)}$$ for all $x \in \cX$ that occur with positive probability (this is all of $\cX$ except for possibly $x = (0,\ldots, 0)$ or $x = (1, \ldots, 1)$ in case $\theta$ is concentrated on $1$ or $0$ respectively).


We claim that $ T:\cX \to \bZ_{\geq 0}$ given by $T(x) = \sum_{i=1}^n x_i$ is a sufficient statistic. Indeed if we fix $x_0 \in \cX = \{0,1\}^n$ and we let $m=Tx_0$, we get that  

\begin{align*} P( \Theta \in B ~|~ T \circ X = m) &= \frac{P( \Theta \in B, T\circ X = m)}{P(T\circ X = m)} 
\\ &= \frac{\sum_{Tx = m} P(\Theta \in B, X = x) }{\sum_{Tx = m}P(X = x)} 
\\ &= \frac{ \sum_{Tx = m} \int_B \theta^{T(x)}(1-\theta)^{n - T(x)} d \mu_{\Theta}(\theta)}{ \sum_{Tx = m} \int_{\Omega} \theta^{T(x)}(1-\theta)^{n - T(x)} d \mu_{\Theta}(\theta)}
\\ &= \frac{ \binom{n}{m} \int_B \theta^{m}(1-\theta)^{n - m} d \mu_{\Theta}(\theta)}{ \binom{n}{m} \int_{\Omega} \theta^{m}(1-\theta)^{n - m} d \mu_{\Theta}(\theta)} \end{align*}

And so we see that this fraction cancels out to the expression for $P(\Theta \in B ~|~ X=x_0)$ that we already obtained.

In fact, the same argument shows that $T$ is still a sufficient statistic if instead of assuming that $P_{\theta}$ is binomial we assume that $P_{\theta}$ is \textit{exchangable}, meaning a distribution on $\cX = \{0,1\}^n$ that is invariant under permutations of the co-ordinates, in this case the same calculation will instead yield $$P( \Theta \in B ~|~ TX = m) = \frac{\sum_{Tx = m} \int_B P_{\theta}(x) d\mu_{\Theta} }{\sum_{Tx = m}  \int_{\Omega} P_{\theta}(x) d\mu_{\Theta} }$$ but since we assume that $P_{\Theta}(x) = P_{\Theta}(x')$ whenever $x$ and $x'$ are permutations of each other, we must have that all the summands are the same and hence we get the same cancellation.
\end{eg}

\begin{remark} Intuitively, sufficiency of a statistic essentially means that if we want to estimate the Posterior distribution of $\theta$ given an observable statistic, we only need to look at the statistic of $T \circ X$ rather than $X$, which may be simpler. In other words, somebody who wishes to estimate $\theta$ (e.g., find a confidence interval) need not know the sample $x$ but just needs to know $Tx$. In this example, the person wishing to estimate $\theta$ needs only to be told a single integer $Tx$, rather than a potentially very long sequence $x \in \{0,1\}^n$ (data compression). 
\end{remark}

The classical definition of sufficiency does not make reference to a prior distribution and is thus not Bayesian.

\begin{mydef} We say that a statistic $T:\cX \to \cT$ is \textit{sufficient in the classical sense} if we have a function $r:\cB \times \cT \to [0,1]$ such that for all $\theta \in \Theta$ and $B \in \cB$ (recall $\cB$ is the Borel $\sigma$-algebra on $\cX$) we have that $$P_{\theta}(B | T=t) = r(B,t) \quad \text{for } T^*P_{\theta}-\text{almost all } t \in \cT.$$

\end{mydef}

\begin{eg} Let us continue from the binomial distribution in Example~\ref{eg: bernoulli bayes setup}. As $\cX = \{0,1\}^n$ is a finite set, we only have to consider the case where $B \subset \cX$ is a singleton $B=\{x_0\}$ for some $x_0 \in \cX$. Now $P_{\theta}(B | T=t) = 0$ if $Tx_0 \neq t$. If $Tx_0 = t$ then we have that $$P_{\theta}(B | T=t) = \frac{P_{\Theta}\{x_0\}}{P_{\theta}\{x \in \cX ~|~ Tx = x_0\}} = \frac{1}{\binom{n}{t}}$$ where the last equality follows from the fact that all $x \in \cX$ with $Tx=t$ have the same probability in a binomial distribution. In fact, the same argument applies to the case where the $P_{\theta}$ are all exchangable. 

Thus we may take $$r(\{x_0\},t) = \mathds{1}(Tx_0 = t) \binom{n}{t}^{-1}$$ and extend $r$ to $\cB \times \bZ$ by additivity. Thus this parametric family is indeed sufficient in the classical sense. 

\end{eg}

\subsection{Equivalence of definitions}

\begin{lemma} A statistic $T:\cX \to \cT$ is sufficient in the Bayesian sense if and only if for all prior distributions $\mu_{\Theta}$ on $\Theta$ and measurable $B \subset \cX$ we have that $$x \mapsto P(\Theta \in B ~|~ x \in X)$$ is almost surely measurely $T^{-1}\cB_T$-measurable (i.e., almost surely equal to a function of $T$).

\end{lemma}

\begin{proof} We prove the non-obvious direction only: Thus suppose that for each measurable $B \subset X$ we have that $x\mapsto P(\Theta \in B ~|~ x \in X) = h_B(Tx)$ where $h_B: \cT \to \bR$ is Borel. We first use apply the law of total probability to the map $T\circ X:\cX \times \Theta \to \cT$ as follows. For $A \subset \cT$ Borel we have that

\begin{align*} P(\Theta \in B, T\circ X \in A) &= \int_{A} P(\Theta \in B| T\circ X = t) dP_{\cT}(t).\end{align*} 

Now we apply the law of total probability to the map $X:\cX \times \Theta \to \cX$ to obtain that 

\begin{align*} P(\Theta \in B, T\circ X \in A) &= P(\Theta \in B , X \in T^{-1}A) \\
&= \int_{T^{-1}A} P(\Theta \in B | X = x)dP_X(x) \\ 
&= \int_{T^{-1}A} h_B(Tx) dP_X(x) \\
&= \int_A h_B(t) dP_{\cT}(t).\end{align*} 

Thus we must have that $P(\Theta \in B | T \circ X = t) = h_B(t)$ holds for almost all $t$. Thus $$P(\Theta \in B ~|~ T \circ X = Tx) = h_B(Tx) = P(\Theta \in B ~|~ X = x)$$ holds for almost all $x \in X$. \end{proof}

\subsection{Fisher-Neyman factorization}

\appendix

\section{Review of Probability}

\section{Borel spaces}

\begin{lemma} Let $X$ be a second-countable metric space. Then there exists an embedding $\iota:X \hookrightarrow Y$ where $Y$ is compact and $\iota$ is a homeomorphism onto its image $\iota(X)$.

\end{lemma}

\begin{proof} Let $u_1,u_2,\ldots $ be a countable dense subset of $X$. Suppose that $d(x,x')\leq 1$ for all $x,x'$. Define $$\iota:X \to [0,1]^{\bN}$$ to be $$\iota(x) = (d(x,u_i))_{i=1}^{\infty}$$ where $Y=[0,1]^{\bN}$ has the product topology, thus is a countable metric space. Continuity and injectivity of $\iota$ is clear. Now suppose that $x_1,x_2,\ldots \in X$ and $x\in X$ are such that $\iota(x_i) \to \iota(x)$ as $i \to \infty$. Fix $\epsilon>0$ and choose $u_j$ so that $d(x,u_j)<\epsilon$. Now for all sufficiently large $i$ we have that $|d(x_i, u_j) - d(x,u_j)| < \epsilon$. This implies that $d(x_i,u_j) < 2\epsilon$. Now $$d(x,x_i) \leq d(x,u_j) + d(u_j,x_i) \leq \epsilon + 2\epsilon  = 3\epsilon.$$ Thus $\iota^{-1}$ is continuous on $\iota(X)$. \end{proof}

\begin{lemma} Let $X$ be a complete metric space, with metric $d_X$, and suppose that $Y \subset X$ is a subset of $X$ such that the subspace topology on $Y$ is induced by a complete metric $d_Y$ on $X$. Then $Y$ is a countable intersection of open subsets of $X$, and is thus a borel subet of $Y$.

\end{lemma}

\begin{proof} We first show that for each $\epsilon>0$ and $y \in Y$ there exists an open set $V_{y,\epsilon}$ of $X$ such that $y \in V$ and $\operatorname{diam}_{d_X}(V_{y,\epsilon}) < \epsilon$ and $\operatorname{diam}_{d_Y}(V_{y,\epsilon} \cap Y) < \epsilon$. To see this, observe since $d_X$ and $d_Y$ induce the same topology on $Y$, there is an open set $\tilde{V}$ of $X$ such that $\tilde{V} \cap Y$ is the ball of radius $\epsilon/2$ around $y$ in the $d_Y$ metric. Hence we can take $$V_{y, \epsilon} = \tilde{V} \cap B_{d_X}(y, \epsilon/2).$$ Now let $U_n = \bigcup_{y \in Y} V_{y,1/n}$ (Remark: to avoid axiom of choice, just take the union of $y$ and all the different sets of the form $\tilde{V}$). Observe now that $Y \subset U_n$ for all $n$. We claim that $$Y = \bigcap_{n = 1}^{\infty} U_n.$$ Thus suppose that $x \in U_n$ for all $n \geq 1$. We will show that there exists a sequence of open subsets $W_1 \supset W_2 \supset \ldots$ of $X$ such that $\operatorname{diam}_{d_X}(W_n) < \frac{1}{n}, \operatorname{diam}_{d_Y}(W_n \cup Y) < \frac{1}{n}$, $x \in W_n$ and $Y \cap W_n$ is non-empty. We can take $W_1 = V_{y_1,1}$ for some $y_1 \in Y$ since $x \in U_1$. Now suppose $W_n$ has been constructed. Thus there exists $\epsilon>0$ so that $B_{d_X}(x, \epsilon) \subset W_n$ as $W_n$ is open and $x \in W_n$. Now take $N>n$ such that $1/N< \epsilon$. Now since $x \in U_N$ there exists $y_{n+1} \in Y$ and $W_{n+1} \subset X$ open so that $y_n, x \in W_{n+1}$ and $\operatorname{diam}_{W_X}(W_{n+1}) < 1/N$, $\operatorname{diam}_{d_Y}(W_{n+1} \cap Y) < 1/N$. Now observe that $W_{n+1} \subset B_{d_X}(x, \epsilon) \subset W_n$ since if $x' \in W_{n+1}$ then $$d(x',x') \leq \operatorname{diam}_{d_Y}(W_{n+1} \cap Y) < 1/N < \epsilon.$$ Thus in summary $y_{n+1} \in W_{n+1}$ is non-empty as desired and $W_{n+1} \subset W_n$ and $$\operatorname{diam}_{d_Y}(W_{n+1} \cap Y) < \frac{1}{N} \leq \frac{1}{n+1}.$$ It now follows that $y_n \to x$ in $d_X$ metric and that $y_1,y_2, \ldots$ is a Cauchy sequence in $Y$. Thus by completeness, we have that $y_n \to y \in Y$ for some $y \in Y$. However $$d_X(x,y) \leq d_X(x,y_n) + d_X(y_n,y) \to 0$$ as $n \to \infty$ since $d_Y(y,y_n) \to 0$ and $d_X$ and $d_Y$ induce the same topology. Thus $x=y \in Y$ as required.\end{proof}

\begin{corol}\label{corol: embedding Polish into compact} Let $X$ be a second countable complete metric space. Then $X$ is homeomorphic to a Borel subset of a compact metric space. In fact, we can take this Borel subset to be a countable intersection of open subsets.

\end{corol}

\begin{mydef} A \textit{standard Borel space} is a measurable space $(X,\cB)$ where $X$ is a second-countable complete metric space and $\cB$ is the Borel $\sigma$-algebra.

\end{mydef}

\subsection{Conditional expectation}

\begin{thm} Let $(X, \cB, \mu)$ be a probability space and let $\cA \subset \cB$ be a sub-$\sigma$-algebra of $\cB$. Then for $f \in L^1(X,\cB,\mu)$ there exists a unique $g \in L^1(X,\cA,\mu)$ such that $$\int_A f d\mu = \int_A g d\mu \quad\text{for all } A\in \cA.$$

\end{thm}

\begin{mydef} We say that the $g \in L^1(X,\cB,\mu)$ the \textit{conditional expectation of $f$ w.r.t. $\cA$}, and denote it by $g = E(f|\cA)$.

\end{mydef}

\begin{proof} If $f \in L^2(X,\cB,\mu)$ then we define $E(f|\cA)$ to be the orthogonal projection of $f$ onto the closed subspace $L^2(X,\cA,\mu)$. Indeed, for $A \in \cA$ we have that $\mathds{1}_A \in L^2(X, \cA,\mu)$ and so $$\int_A f d\mu = \langle f, \mathds{1}_A \rangle = \langle E(f|\cA), \mathds{1}_A \rangle = \int_A  E(f|\cA) d\mu$$ as required. We will now extend $E(\cdot |\cA)$ to $L^1$. By density of $L^2$ in $L^1$ it is enough to show that $E(\cdot |\cA)$ is a bounded operator w.r.t. to the $L^1$ norm on $L^2$. For $f \in L^2(X,\cB,\mu)$ let $g \in L^{\infty}(X,\cA,\mu)$ be such that $|E(f|\cA)| = gE(f|\cA)$ and $\|g\|_{\infty} \leq 1$. Thus $$\|E(f|\cA)\|_{1} = \int_X gE(f|\cA) d\mu = \langle E(f|\cA), \overline{g} \rangle = \langle f, \overline{g} \rangle = \int_X g f d\mu \leq \|f\|_1.$$ Thus $E(\cdot |\cA)$ has $L^1$ operator norm at most $1$ and thus can be extended to $L^1(X,\cB,\mu) \to L^1(X,\cA\mu)$.

For uniqueness, suppose that $g_1,g_2 \in L^1(X,\cA,\mu)$ are real valued functions such that $$\int_A g_1 d\mu = \int_A g_2 d\mu \quad \text{for all } A \in \cA.$$ Let $$A_{\epsilon} = \{x \in X ~|~ g_1(x) \leq g_2(x) - \epsilon \}$$ and observe that $A_{\epsilon} \in \cA$. Thus $$\int_A g_2 = \int_A g_1 d\mu \leq \int_A g_2 d\mu - \epsilon \mu(A)$$ which implies that $\mu(A_{\epsilon}) = 0$. Thus $g_1 \geq g_2$ almost everywhere. By symmetry, $g_1=g_2$ almost everywhere. For $g_1,g_2$ complex valued uniqueness also follows by equating real and imaginary parts. 
\end{proof}

\begin{mydef} Let $T:X \to Y$ be a measurable map between measure spaces $(X,\cB_X)$ and $(Y,\cB_Y)$. We define $T^{-1}\cB_Y = \{ T^{-1}B ~|~ B \in \cB_Y\}$.

\end{mydef}

\begin{prop} Suppose that $(X,\cB_X, \mu_X)$ is a probability space, $(Y,\cB_Y)$ is a measurable space $T:X \to Y$ is measurable. Then any function $f:X \to \bR$ that is $T^{-1}\cB_Y$ measurable (that is, $f^{-1}B \in T^{-1}\cB_Y$ for all Borel $B \subset \bR$) can be written as $f = g \circ T$ for some measurable $g:Y \to \bR$. 

In other words, if we let $\mu_Y = T^* \mu_X$ be the push-forward measure then we have an isometric embedding $$L^p(Y, \cB_Y, \mu_Y) \hookrightarrow L^p(X,\cB_X, \mu_X)$$ given by $g \mapsto g \circ T$ whose image is exactly $L^p(\cX, T^{-1}\cA, \mu_X)$.
\end{prop}

\begin{proof} If $f = \mathds{1}_A$ for some $A \in T^{-1}\cB_Y$ then $A = T^{-1}B$ for some $B \in \cB_Y$ and so we can take $g = \mathds{1}_B$. For general $f$ note that we can write it as a pointwise limit of sums of indicator functions. \end{proof}

This proposition motivates us to define conditional expectation with respect to a random variable (or factor map). 

\begin{mydef} Let $(X, \cB, \mu)$ be a probability space and suppose that $T:X \to Y$ is measurable where $(Y, \cB_Y)$ is a measure space (i.e., a random variable). Then for $f \in L^p(\cX,\cB,\mu)$ we define a function on $Y \to \bC$ denoted $y \mapsto E(f| T=y)$ that is characterized for $\mu$-almost all $x$ by  $$E(f|\cT^{-1}\cB_Y)(x) = E(f| T=y),$$ where $y = Tx$. In other words, the map $E(f|\cT^{-1}\cB_Y) \in L^p(\cX,T^{-1}\cB_Y,\mu)$ is identified with a function of $Y$ using the isometry in the proposition above and this function at $y \in Y$ is  $E(f| T=y)$. Note that $E(f|T=y)$ is uniquely defined for $T^*\mu_X$-almost all $y \in Y$.

\end{mydef}

Note that $E(f|T=y)$ can be characterized by the property that $$\int_A E(f|T=y) d\mu_Y(y) = \int_{T^{-1}A} f(x) d\mu_X(x) \quad \text{for all } A \in \cB_Y.$$ 

\subsection{Conditional Probability}

\begin{thm}[Conditional probability] Let $X$ be a second-countable complete metric space with Borel $\sigma$-algebra $\cB$ and suppose $\cA \subset \cB$ is a sub $\sigma$-algebra. Let $\mu$ be a probability measure on $(X,\cB)$. Then there exists $X' \in \cA$ with $\mu(X')=1$ and a family of probability measures $\mu_x^{\cA}$ for $x \in X'$ such that the following holds.

\begin{itemize}
	\item For $\cB$-measurable functions $f:X \to \bC$ such that $\|f\|_{\infty} < \infty$, we have that the mapping $X' \to \bC$ given by $$x \mapsto \int_X f(u) d\mu_x^{\cA} (u)$$ is $\cA$-measurable. The same is true if we replace $\|f\|_{\infty} < \infty$ with $f(x) \in \bR_{\geq 0}$ for all $x \in X$.
	\item For $f: X \to \bC$ with $\|f\|_1 < \infty$ we have that\footnote{The right hand side may be undefined on a null set of $x$.}  $$E(f|\cA)(x) = \int_X f(u) d\mu_x^{\cA} (u) \quad \text{ for almost all } x \in X.$$

\end{itemize}

\end{thm}

\begin{mydef} We call such a family of measures $\mu^{\cA}_x$ \textit{conditional probabilities of $\mu$ w.r.t $\cA$}.

\end{mydef}

\begin{proof} Using Corollary~\ref{corol: embedding Polish into compact} we may assume $X$ is compact. Thus there exists a countable dense $\bQ$-vector subspace $V$ of $C(X)$ such that $1 \in V$. Now for each $f \in V$ choose a geniune function (not an equivalence class of functions) $g_f:X \to \bC$ such that $$g_f(x) = E(f|\cA)(x)$$ and $$\|g_f\|_{\infty} \leq \|f\|_{\infty}.$$

 Now for any $\alpha,\beta \in \bQ$ and $f_1,f_2 \in V$ we have that $$g_{\alpha f_1 + \beta f_2}(x) = \alpha g_{f_1}(x) + \beta g_{f_2}(x)\quad \text{ for almost all } x \in X.$$ We also have $g_1(x) = 1$ and $g_f(x) \geq 0$ for positive $f \in V$ (i.e, $f\geq 0$) for almost all $x \in X$. Thus we may choose $X' \subset X$ to be a set with $\mu(X')=1$ on which all of thse properties hold (as there are only countably many equations). Thus for each $x \in X'$ we have a linear function $\Lambda'_x: V \to \bC$ given by $$\Lambda'_x (f)= g_f(x).$$ Note that $\Lambda'_x$ is uniformly continuous (since $\|g_f\|_{\infty} \leq \|f\|_{\infty}$) and thus it extends to a linear functional $\Lambda_x:C(X) = \overline{V} \to \bC$ which also satisfies that $$|\Lambda_x (f)| \leq \|f\|_{\infty}.$$ We wish to apply the Riesz representation theorem, thus it remains to show that $\Lambda_x$ is a positive linear functional. Now suppose $f \in C(X)$ is positive. Then there exists $f_i \in V$ such that $f_i \to f$ uniformly. Now fix a positive integer $q$ and let $\epsilon = 1/q$. Then $\Lambda_x(\epsilon) = \epsilon$ since $g_1=1$ and also $f_i + \epsilon \geq 0$ for $i$ sufficiently large (as for $i$ sufficiently large we have $\|f_i - f\|_{\infty} < \epsilon$ and $f$ is positive). As $f_i + \epsilon \in V$ we have that $$\Lambda_x(f_i) + \epsilon = g_{f_i + \epsilon}(x) \geq 0.$$ Thus $$\Lambda_f(x) \geq \Lambda_x(f_i) - \epsilon> -2\epsilon.$$ This shows that $\Lambda_x(f) \geq 0$ and thust $\Lambda_x$ is a positive linear continuous functional on $C(X)$ with $\Lambda_x(1)=1$. Thus there exists a measure $\mu_x$ such that $$\Lambda_x(f) = \int_X f(u) d\mu_x(u) \quad \text{for all } f \in C(X), x \in X'.$$ In particular this means that $$ E(f|\cA)(x) = \int_X f(u) d\mu_x(u) \quad \text{for all } f \in V.$$

This means that \begin{align}\label{eqn: measurable integrand} x \mapsto \int_{X'} f(u) d\mu_x(u) \text{ is $\cA$-measurable on $X'$} \end{align} and \begin{align}\label{eqn: cond probability on f} \int_{A} \int_{X'} f(u) d\mu_x(u) d\mu(x) = \int_A f(x) d\mu(x) \quad\text{ for all } A \in \cA \end{align} holds for all $f \in V$. By density of $V$ in $C(X)$, this also holds for all $f \in C(X)$ and the integrand is indeed $\cA$ measurable (it is a pointwise limit of $\cA$ measurable functions). Since the indicator function of an open set is a monotonic limit of continuous functions, we have that (\ref{eqn: measurable integrand}) and (\ref{eqn: cond probability on f}) holds for $f = \mathds{1}_U$ for all $U \subset X$ open. It also holds for closed sets by linearity. Finally, observe that it also holds for an intersection $U \cap C$ where $U$ is open and $C$ is closed since such a set is a countable nested intersection of open sets.

Now let $$\cR = \{ \cup_{i=1}^n U_i \cap C_i ~|~ U_1, \ldots, U_n \text{ are open and } C_1, \ldots, C_n \text{ are closed.} \}$$ be the set of all finite unions of sets that can be written as an intersection of an open set and a closed set. We claim that any element in $\cR$ can be written as a finite disjoint union of an intersection of an open set with a closed set. Indeed, any such finite union belongs to a finite $\sigma$-algebra generated by such sets but any atom of this $\sigma$-algebra must take this form. It now follows that  (\ref{eqn: measurable integrand}) and (\ref{eqn: cond probability on f}) holds for $f = \mathds{1}_R$ for all $R \in \cR$. By the monotone convergence theorem, these also hold on the monotone class generated by $\cR$. Thus by the monotone class lemma, it holds on all Borel sets since $\cR$ is an algebra. Any measurable $f:X \to \bR_{\geq 0}$ is a monotonic limit of simple functions and so it holds for such $f$ too, which completes the proof for such $f$. The theorem now follows for general $f:X \to \bC$ with $\|f\|_1$ as desired by linearity. \end{proof}

We now show that the conditional probabilities are almost surely unique and provide some characterizations that are practial to use. In the following proposition, we fix $X,\cA, \mu$ as as in Theorem~\ref{thm: cond prop existence}.

\begin{prop} If $\mu_x^{\cA}$ and $\nu_x^{\cA}$ are two conditional probabilities, for some $x \in X' \in \cA$ of full measure, then $\mu_x^{\cA} = \nu_x^{\cA}$ for $\mu$-almost all $x \in X$.

\end{prop}

\begin{proof} We for each $f \in C(X)$ we have that $$\int_X f(u) d\mu_x^{\cA} (u) = E(f|\cA)(x) = \int_X f(u) d\nu_x^{\cA} (u)$$ holds for all $x \in X_f \in \cA$ with $\mu(X_f) = 1$. Applying this to a countable dense set of $f \in C(X)$ we can find a single set $X' \in \cA$ with $\mu(X'')=1$ so that this identity holds for all continuous functions $f \in C(X)$. It follows by uniqueness in Riesz-representation theorem that $\mu_x^{\cA} = \nu_x^{\cA}$ for all $x \in X''$. 

\end{proof}

\subsection{Conditional probability with respect to a random variable}

In this subsection, let $\cA$ is a countable generated $\sigma$-algebra of $\cB$ where $(X,\cB)$ is standard Borel. So $\cA= \sigma(\{A_1,A_2,\ldots, \})$ where $A_1,A_2,\ldots \in \cA$. For each $x \in X$ we can define the atom $$[x]_{\cA} = \bigcap_{i, x \in A_i} A_i \cap \bigcap_{j, x \notin A_j} (X \setminus A_j)$$ containing $x$.

\begin{prop} For $x \in X$ we have that $[x]$ is the smallest subset of $\cA$ containing $x$.

\end{prop}

\begin{proof} We must show that for each $A \in \cA$ we have either $[x] \subset \cA$ or $[x] \cap \cA = \emptyset$. This is true for each $A_i$ and so one must show that the sets in $\cA$ satisfying this property forms a $\sigma$-algebra. \end{proof}

\begin{prop} There is a set $X'' \subset X$ with $X'' \in \cA$ and $\mu(X'')=1$ such that for any $x_1,x_2 \in X''$ we have that $\mu_{x_1}^{\cA} = \mu_{x_2}^{\cA}$ if and only if $[x_1] = [x_2]$. Moreover, $\mu_x^{\cA}([x]) = 1$ for $x \in X''$.

\end{prop}

\begin{proof} Let $\mu_x = \mu_x^{\cA}$ for $x \in X'$ be a set of conditional probabilities, where $X' \in \cA$ and $\mu(X')=1$. For almost all $x \in X'$ we have that $$\mu_x(A_i) = \int \mathds{1}_{A_i} d\mu_x = E(\mathds{1}_{A_i}|\cA)(x) = \mathds{1}_{A_i}(x).$$ Say this occurs on $X'' \subset X'$ with $\mu(X'')=1$. Note that since all terms are $\cA$-measurable, we have $X'' \in \cA$. Now let $x_1,x_2 \in X''$. If $\mu_{x_1} = \mu_{x_2}$ then $\mathds{1}_{A_i}(x_1) = \mathds{1}_{A_i}(x_2)$ for all $i$ and thus $[x_1] = [x_2]$. Conversely, suppose that $[x_1]=[x_2]$. Thus any $\cA$-measurable function must have equal values on $x_1$ and $x_2$. But for any $f \in C(X)$ we have that $$x \mapsto \int f d\mu_x$$ is $\cA$-measurable. Thus $$\int f d\mu_{x_1} = \int f d \mu_{x_2}$$ holds for all $f \in C(X)$ and thus $\mu_{x_1} = \mu_{x_2}$. \end{proof}

\begin{thm}[Law of total probability] Suppose that $T:X\to Y$ is a measurable map between standard Borel spaces and let $\cB_T := T^{-1}\cB_Y$ be the induced $\sigma$-algebra. There exists Borel $Y' \subset Y$ and probability measures $\mu_y$ on $X$, for each $y \in Y'$, such that the following hold.

\begin{itemize}
	\item $\mu(T^{-1}Y') = 1$
	\item For all $y \in Y'$, the measure $\mu_y$ is supported on $T^{-1}(\{y \})$.
	\item For all $x \in T^{-1}(Y')$ we have that $\mu_{Tx} = \mu_{x}^{\cB_T}$. That is the family $\mu_{Tx}$ for $x \in T^{-1}(Y')$ is a family of conditional probability measures for the $\sigma$-algebra $\cB_T$.
\end{itemize}

Moreover, for bounded Borel $f: X \to \bC$ and $B \subset Y$ Borel we have that $$\int_{T^{-1}B} f(x) d\mu(x) = \int_{B} \left( \int_X f(u) d\mu_y(u) \right) d(T^*\mu)(y).$$

\end{thm}

\begin{proof} The induced $\sigma$-algebra $\cB_T := T^{-1}\cB_Y$ is countably generated and its atoms are precisely the fibers $$T^{-1}(\{y\}) \quad \text{for } y \in Y.$$ In other words, $$[x]_{\cB_T} = T^{-1}(\{ Tx \}) \quad \text{for } x \in X.$$ We thus have by the above proposition that the conditional measure $\mu_{x}^{\cB_T}$ is supported on the fiber $T^{-1}(\{Tx\})$ and only depends on the fiber. More precisely, we have a family of measures $\mu_y$ on $X$, $y \in Y'$ for some Borel $Y' \subset Y$ such that $\mu_y$ is supported on $T^{-1}(\{y \})$ and such that for bounded Borel $f:X \to \bC$ we have that $$x \mapsto \int_X f(u) d\mu_{Tx}(u)$$ is $\cB_T$ measurable and $$\int_A f(x) d\mu(x) = \int_A \int f(u) d\mu_{Tx}(u) d\mu(x) \quad \text{ for } A \in \cB_T.$$ Now for $B \subset Y'$ we have that $A= T^{-1}B \in \cB_T$ and $$\int_A g(Tx) d\mu(x) = \int_B g(y) d (T^*\mu)(y)$$ for $g:X \to \bC$ Borel and bounded. \end{proof}

It is customary to define $$\mu(B ~|~ T = y) := \mu_y(B)$$ and $$E(f | T=y) := E(f|\cB_T)(x)$$ for $y = Tx$. The law of total probability then says $$\mu(A \cap T^{-1}B) = \int_B \mu(A | T=y) d\mu_Y(y)  $$ for any Borel $A \subset X, B \subset Y$ where $\mu_Y = T^*\mu$.

\section{Useful results}

\begin{lemma} Suppose that $\nu$ is a $\sigma$-finite positive measure on $X$. Then there exists a probability measure $\nu'$ such that $\nu \ll \nu'$ and $\nu' \ll \nu$ (i.e., $\nu(A) = 0$ if and only if $\nu'(A) = 0$).

\end{lemma}

\begin{proof} We only need to consider the case $\nu(X) = \infty$. There exists a countable partition $X_1, \ldots$ of $X$ such that $0< \nu(X_i) = d_i < \infty$. Now define $$\nu'(A) = \sum_{i=1}^{\infty} 2^{-i}d_i^{-1} \nu(A \cap X_i).$$

\end{proof}

\begin{lemma} \label{lemma: countable sum dominates} Let $\nu$ be a $\sigma$-finite positive measure on $X$ and ;et $\cP$ be a family of probability measures on $X$ such that $P \ll \nu$ for all $P \in \cP$. Then there exists $a_1,a_2,\ldots \in [0,1]$ with $\sum_{i} a_i = 1$ and $P_1,P_2, \ldots \in \cP$ such that $$P \ll \sum_{i=1}^{\infty} a_i P_i \quad \text{for all } P \in \cP.$$

\end{lemma}

\begin{proof} By the previous Lemma, we only need to focus on the case that $\nu(X)=1$.  Let $$\cQ = \{ \sum_{i=1}^{\infty} a_i P_i ~|~ \sum_{i} a_i = 1, a_i \geq 0, P_i \in \cP \}$$ be the set of countable convex combinations of measures in $\cP$. Observe that $Q \ll \nu$ for all $Q \in \cQ$. For $Q \in \cQ$ let $f_Q = dQ/d\nu$ be the density of $Q$ and let $$S_Q = \{ x \in X ~|~ f_Q(x)>0 \}.$$ We remark that $f_Q$ and $S_Q$ are almost surely well defined. Now let $c = \sup_{Q \in \cQ} \nu(S_Q) \leq 1$. Let $Q_1,Q_2,\ldots \in \cQ$ be such that $\nu(S_Q) \to c$. Now there exists $Q_0$ such that $$S_{Q_0} = \cup_{i=1}^{\infty} S_{Q_i}$$ and thus $\nu(S_{Q_0}) = c$. We claim that $P \ll Q_0$ for all $P \in \cP$, thus completing the proof. Suppose it were not the case. Then there exists $A \subset X$ such that $P(A) >0$ but $Q_0(A)=0$. Now let $$Q' = \frac{1}{2}\left( Q_0 + P \right).$$ We have that $Q_0 \ll Q'$ and so $S_{Q_0} \subset S_{Q'}$. Since $Q_0(A)=0$ we have that $\nu(A \cap S_{Q_0}) = 0$. On the other hand $$0<P(A) = \int_{A} f_P d\nu \leq 2\int_A f_{Q'} d\nu$$ and so this means that there is $A' \subset A$ with $\nu(A')>0$ and $f_{Q'}(x) >0$ for $x \in A'$. Thus $$S_{Q_0} \cup A' \subset S_{Q'}$$ and so $$\nu(S_{Q'}) \geq \nu(S_{Q_0}) + \nu(A') > c,$$ which is a contradiction. \end{proof}
\end{document}
